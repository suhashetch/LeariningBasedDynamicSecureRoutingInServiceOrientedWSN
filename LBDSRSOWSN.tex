
%% bare_jrnl.tex
%% V1.3
%% 2007/01/11
%% by Michael Shell
%% see http://www.michaelshell.org/
%% for current contact information.
%%
%% This is a skeleton file demonstrating the use of IEEEtran.cls
%% (requires IEEEtran.cls version 1.7 or later) with an IEEE journal paper.
%%
%% Support sites:
%% http://www.michaelshell.org/tex/ieeetran/
%% http://www.ctan.org/tex-archive/macros/latex/contrib/IEEEtran/
%% and
%% http://www.ieee.org/



% *** Authors should verify (and, if needed, correct) their LaTeX system  ***
% *** with the testflow diagnostic prior to trusting their LaTeX platform ***
% *** with production work. IEEE's font choices can trigger bugs that do  ***
% *** not appear when using other class files.                            ***
% The testflow support page is at:
% http://www.michaelshell.org/tex/testflow/


%%*************************************************************************
%% Legal Notice:
%% This code is offered as-is without any warranty either expressed or
%% implied; without even the implied warranty of MERCHANTABILITY or
%% FITNESS FOR A PARTICULAR PURPOSE! 
%% User assumes all risk.
%% In no event shall IEEE or any contributor to this code be liable for
%% any damages or losses, including, but not limited to, incidental,
%% consequential, or any other damages, resulting from the use or misuse
%% of any information contained here.
%%
%% All comments are the opinions of their respective authors and are not
%% necessarily endorsed by the IEEE.
%%
%% This work is distributed under the LaTeX Project Public License (LPPL)
%% ( http://www.latex-project.org/ ) version 1.3, and may be freely used,
%% distributed and modified. A copy of the LPPL, version 1.3, is included
%% in the base LaTeX documentation of all distributions of LaTeX released
%% 2003/12/01 or later.
%% Retain all contribution notices and credits.
%% ** Modified files should be clearly indicated as such, including  **
%% ** renaming them and changing author support contact information. **
%%
%% File list of work: IEEEtran.cls, IEEEtran_HOWTO.pdf, bare_adv.tex,
%%                    bare_conf.tex, bare_jrnl.tex, bare_jrnl_compsoc.tex
%%*************************************************************************

% Note that the a4paper option is mainly intended so that authors in
% countries using A4 can easily print to A4 and see how their papers will
% look in print - the typesetting of the document will not typically be
% affected with changes in paper size (but the bottom and side margins will).
% Use the testflow package mentioned above to verify correct handling of
% both paper sizes by the user's LaTeX system.
%
% Also note that the "draftcls" or "draftclsnofoot", not "draft", option
% should be used if it is desired that the figures are to be displayed in
% draft mode.
%
\documentclass[journal]{IEEEtran}
%
% If IEEEtran.cls has not been installed into the LaTeX system files,
% manually specify the path to it like:
% \documentclass[journal]{../sty/IEEEtran}





% Some very useful LaTeX packages include:
% (uncomment the ones you want to load)


% *** MISC UTILITY PACKAGES ***
%
%\usepackage{ifpdf}
% Heiko Oberdiek's ifpdf.sty is very useful if you need conditional
% compilation based on whether the output is pdf or dvi.
% usage:
% \ifpdf
%   % pdf code
% \else
%   % dvi code
% \fi
% The latest version of ifpdf.sty can be obtained from:
% http://www.ctan.org/tex-archive/macros/latex/contrib/oberdiek/
% Also, note that IEEEtran.cls V1.7 and later provides a builtin
% \ifCLASSINFOpdf conditional that works the same way.
% When switching from latex to pdflatex and vice-versa, the compiler may
% have to be run twice to clear warning/error messages.






% *** CITATION PACKAGES ***
%
%\usepackage{cite}
% cite.sty was written by Donald Arseneau
% V1.6 and later of IEEEtran pre-defines the format of the cite.sty package
% \cite{} output to follow that of IEEE. Loading the cite package will
% result in citation numbers being automatically sorted and properly
% "compressed/ranged". e.g., [1], [9], [2], [7], [5], [6] without using
% cite.sty will become [1], [2], [5]--[7], [9] using cite.sty. cite.sty's
% \cite will automatically add leading space, if needed. Use cite.sty's
% noadjust option (cite.sty V3.8 and later) if you want to turn this off.
% cite.sty is already installed on most LaTeX systems. Be sure and use
% version 4.0 (2003-05-27) and later if using hyperref.sty. cite.sty does
% not currently provide for hyperlinked citations.
% The latest version can be obtained at:
% http://www.ctan.org/tex-archive/macros/latex/contrib/cite/
% The documentation is contained in the cite.sty file itself.






% *** GRAPHICS RELATED PACKAGES ***
%
\ifCLASSINFOpdf
  % \usepackage[pdftex]{graphicx}
  % declare the path(s) where your graphic files are
  % \graphicspath{{../pdf/}{../jpeg/}}
  % and their extensions so you won't have to specify these with
  % every instance of \includegraphics
  % \DeclareGraphicsExtensions{.pdf,.jpeg,.png}
\else
  % or other class option (dvipsone, dvipdf, if not using dvips). graphicx
  % will default to the driver specified in the system graphics.cfg if no
  % driver is specified.
  % \usepackage[dvips]{graphicx}
  % declare the path(s) where your graphic files are
  % \graphicspath{{../eps/}}
  % and their extensions so you won't have to specify these with
  % every instance of \includegraphics
  % \DeclareGraphicsExtensions{.eps}
\fi
% graphicx was written by David Carlisle and Sebastian Rahtz. It is
% required if you want graphics, photos, etc. graphicx.sty is already
% installed on most LaTeX systems. The latest version and documentation can
% be obtained at: 
% http://www.ctan.org/tex-archive/macros/latex/required/graphics/
% Another good source of documentation is "Using Imported Graphics in
% LaTeX2e" by Keith Reckdahl which can be found as epslatex.ps or
% epslatex.pdf at: http://www.ctan.org/tex-archive/info/
%
% latex, and pdflatex in dvi mode, support graphics in encapsulated
% postscript (.eps) format. pdflatex in pdf mode supports graphics
% in .pdf, .jpeg, .png and .mps (metapost) formats. Users should ensure
% that all non-photo figures use a vector format (.eps, .pdf, .mps) and
% not a bitmapped formats (.jpeg, .png). IEEE frowns on bitmapped formats
% which can result in "jaggedy"/blurry rendering of lines and letters as
% well as large increases in file sizes.
%
% You can find documentation about the pdfTeX application at:
% http://www.tug.org/applications/pdftex





% *** MATH PACKAGES ***
%
\usepackage[cmex10]{amsmath}
% A popular package from the American Mathematical Society that provides
% many useful and powerful commands for dealing with mathematics. If using
% it, be sure to load this package with the cmex10 option to ensure that
% only type 1 fonts will utilized at all point sizes. Without this option,
% it is possible that some math symbols, particularly those within
% footnotes, will be rendered in bitmap form which will result in a
% document that can not be IEEE Xplore compliant!
%
% Also, note that the amsmath package sets \interdisplaylinepenalty to 10000
% thus preventing page breaks from occurring within multiline equations. Use:
%\interdisplaylinepenalty=2500
% after loading amsmath to restore such page breaks as IEEEtran.cls normally
% does. amsmath.sty is already installed on most LaTeX systems. The latest
% version and documentation can be obtained at:
% http://www.ctan.org/tex-archive/macros/latex/required/amslatex/math/





% *** SPECIALIZED LIST PACKAGES ***
%
%\usepackage{algorithmic}
% algorithmic.sty was written by Peter Williams and Rogerio Brito.
% This package provides an algorithmic environment fo describing algorithms.
% You can use the algorithmic environment in-text or within a figure
% environment to provide for a floating algorithm. Do NOT use the algorithm
% floating environment provided by algorithm.sty (by the same authors) or
% algorithm2e.sty (by Christophe Fiorio) as IEEE does not use dedicated
% algorithm float types and packages that provide these will not provide
% correct IEEE style captions. The latest version and documentation of
% algorithmic.sty can be obtained at:
% http://www.ctan.org/tex-archive/macros/latex/contrib/algorithms/
% There is also a support site at:
% http://algorithms.berlios.de/index.html
% Also of interest may be the (relatively newer and more customizable)
% algorithmicx.sty package by Szasz Janos:
% http://www.ctan.org/tex-archive/macros/latex/contrib/algorithmicx/




% *** ALIGNMENT PACKAGES ***
%
%\usepackage{array}
% Frank Mittelbach's and David Carlisle's array.sty patches and improves
% the standard LaTeX2e array and tabular environments to provide better
% appearance and additional user controls. As the default LaTeX2e table
% generation code is lacking to the point of almost being broken with
% respect to the quality of the end results, all users are strongly
% advised to use an enhanced (at the very least that provided by array.sty)
% set of table tools. array.sty is already installed on most systems. The
% latest version and documentation can be obtained at:
% http://www.ctan.org/tex-archive/macros/latex/required/tools/


%\usepackage{mdwmath}
%\usepackage{mdwtab}
% Also highly recommended is Mark Wooding's extremely powerful MDW tools,
% especially mdwmath.sty and mdwtab.sty which are used to format equations
% and tables, respectively. The MDWtools set is already installed on most
% LaTeX systems. The lastest version and documentation is available at:
% http://www.ctan.org/tex-archive/macros/latex/contrib/mdwtools/


% IEEEtran contains the IEEEeqnarray family of commands that can be used to
% generate multiline equations as well as matrices, tables, etc., of high
% quality.


%\usepackage{eqparbox}
% Also of notable interest is Scott Pakin's eqparbox package for creating
% (automatically sized) equal width boxes - aka "natural width parboxes".
% Available at:
% http://www.ctan.org/tex-archive/macros/latex/contrib/eqparbox/





% *** SUBFIGURE PACKAGES ***
%\usepackage[tight,footnotesize]{subfigure}
% subfigure.sty was written by Steven Douglas Cochran. This package makes it
% easy to put subfigures in your figures. e.g., "Figure 1a and 1b". For IEEE
% work, it is a good idea to load it with the tight package option to reduce
% the amount of white space around the subfigures. subfigure.sty is already
% installed on most LaTeX systems. The latest version and documentation can
% be obtained at:
% http://www.ctan.org/tex-archive/obsolete/macros/latex/contrib/subfigure/
% subfigure.sty has been superceeded by subfig.sty.



%\usepackage[caption=false]{caption}
%\usepackage[font=footnotesize]{subfig}
% subfig.sty, also written by Steven Douglas Cochran, is the modern
% replacement for subfigure.sty. However, subfig.sty requires and
% automatically loads Axel Sommerfeldt's caption.sty which will override
% IEEEtran.cls handling of captions and this will result in nonIEEE style
% figure/table captions. To prevent this problem, be sure and preload
% caption.sty with its "caption=false" package option. This is will preserve
% IEEEtran.cls handing of captions. Version 1.3 (2005/06/28) and later 
% (recommended due to many improvements over 1.2) of subfig.sty supports
% the caption=false option directly:
%\usepackage[caption=false,font=footnotesize]{subfig}
%
% The latest version and documentation can be obtained at:
% http://www.ctan.org/tex-archive/macros/latex/contrib/subfig/
% The latest version and documentation of caption.sty can be obtained at:
% http://www.ctan.org/tex-archive/macros/latex/contrib/caption/




% *** FLOAT PACKAGES ***
%
%\usepackage{fixltx2e}
% fixltx2e, the successor to the earlier fix2col.sty, was written by
% Frank Mittelbach and David Carlisle. This package corrects a few problems
% in the LaTeX2e kernel, the most notable of which is that in current
% LaTeX2e releases, the ordering of single and double column floats is not
% guaranteed to be preserved. Thus, an unpatched LaTeX2e can allow a
% single column figure to be placed prior to an earlier double column
% figure. The latest version and documentation can be found at:
% http://www.ctan.org/tex-archive/macros/latex/base/



%\usepackage{stfloats}
% stfloats.sty was written by Sigitas Tolusis. This package gives LaTeX2e
% the ability to do double column floats at the bottom of the page as well
% as the top. (e.g., "\begin{figure*}[!b]" is not normally possible in
% LaTeX2e). It also provides a command:
%\fnbelowfloat
% to enable the placement of footnotes below bottom floats (the standard
% LaTeX2e kernel puts them above bottom floats). This is an invasive package
% which rewrites many portions of the LaTeX2e float routines. It may not work
% with other packages that modify the LaTeX2e float routines. The latest
% version and documentation can be obtained at:
% http://www.ctan.org/tex-archive/macros/latex/contrib/sttools/
% Documentation is contained in the stfloats.sty comments as well as in the
% presfull.pdf file. Do not use the stfloats baselinefloat ability as IEEE
% does not allow \baselineskip to stretch. Authors submitting work to the
% IEEE should note that IEEE rarely uses double column equations and
% that authors should try to avoid such use. Do not be tempted to use the
% cuted.sty or midfloat.sty packages (also by Sigitas Tolusis) as IEEE does
% not format its papers in such ways.


%\ifCLASSOPTIONcaptionsoff
%  \usepackage[nomarkers]{endfloat}
% \let\MYoriglatexcaption\caption
% \renewcommand{\caption}[2][\relax]{\MYoriglatexcaption[#2]{#2}}
%\fi
% endfloat.sty was written by James Darrell McCauley and Jeff Goldberg.
% This package may be useful when used in conjunction with IEEEtran.cls'
% captionsoff option. Some IEEE journals/societies require that submissions
% have lists of figures/tables at the end of the paper and that
% figures/tables without any captions are placed on a page by themselves at
% the end of the document. If needed, the draftcls IEEEtran class option or
% \CLASSINPUTbaselinestretch interface can be used to increase the line
% spacing as well. Be sure and use the nomarkers option of endfloat to
% prevent endfloat from "marking" where the figures would have been placed
% in the text. The two hack lines of code above are a slight modification of
% that suggested by in the endfloat docs (section 8.3.1) to ensure that
% the full captions always appear in the list of figures/tables - even if
% the user used the short optional argument of \caption[]{}.
% IEEE papers do not typically make use of \caption[]'s optional argument,
% so this should not be an issue. A similar trick can be used to disable
% captions of packages such as subfig.sty that lack options to turn off
% the subcaptions:
% For subfig.sty:
% \let\MYorigsubfloat\subfloat
% \renewcommand{\subfloat}[2][\relax]{\MYorigsubfloat[]{#2}}
% For subfigure.sty:
% \let\MYorigsubfigure\subfigure
% \renewcommand{\subfigure}[2][\relax]{\MYorigsubfigure[]{#2}}
% However, the above trick will not work if both optional arguments of
% the \subfloat/subfig command are used. Furthermore, there needs to be a
% description of each subfigure *somewhere* and endfloat does not add
% subfigure captions to its list of figures. Thus, the best approach is to
% avoid the use of subfigure captions (many IEEE journals avoid them anyway)
% and instead reference/explain all the subfigures within the main caption.
% The latest version of endfloat.sty and its documentation can obtained at:
% http://www.ctan.org/tex-archive/macros/latex/contrib/endfloat/
%
% The IEEEtran \ifCLASSOPTIONcaptionsoff conditional can also be used
% later in the document, say, to conditionally put the References on a 
% page by themselves.





% *** PDF, URL AND HYPERLINK PACKAGES ***
%
%\usepackage{url}
% url.sty was written by Donald Arseneau. It provides better support for
% handling and breaking URLs. url.sty is already installed on most LaTeX
% systems. The latest version can be obtained at:
% http://www.ctan.org/tex-archive/macros/latex/contrib/misc/
% Read the url.sty source comments for usage information. Basically,
% \url{my_url_here}.





% *** Do not adjust lengths that control margins, column widths, etc. ***
% *** Do not use packages that alter fonts (such as pslatex).         ***
% There should be no need to do such things with IEEEtran.cls V1.6 and later.
% (Unless specifically asked to do so by the journal or conference you plan
% to submit to, of course. )


% correct bad hyphenation here
\hyphenation{op-tical net-works semi-conduc-tor}


\begin{document}
%
% paper title
% can use linebreaks \\ within to get better formatting as desired
\title{Learning Based Dynamic Secure Load Balancing In Service Oriented Wireless Sensor Networks}
%
%
% author names and IEEE memberships
% note positions of commas and nonbreaking spaces ( ~ ) LaTeX will not break
% a structure at a ~ so this keeps an author's name from being broken across
% two lines.
% use \thanks{} to gain access to the first footnote area
% a separate \thanks must be used for each paragraph as LaTeX2e's \thanks
% was not built to handle multiple paragraphs
%

\author{Lata~BT,~\IEEEmembership{}
        Sumukha~TV,~\IEEEmembership{Student Member,~IEEE,}
        Suhas~H,~\IEEEmembership{Student Member,~IEEE,}
	 Shaila~K,
	 Venugopal~KR,~\IEEEmembership{Life Member,~IEEE,}
       LM~Patnaik% <-this % stops a space
\thanks{Lata BT is with the Department
of Computer Science and Engineering, University Visvesvaraya College of Engineering,
Bangalore, 560001 India e-mail: lata\_bt@yahoo.com.}% <-this % stops a space
\thanks{Sumukha TV and Suhas H are with UVCE.}% <-this % stops a space
\thanks{Manuscript received April 19, 2005; revised January 11, 2007.}}

% note the % following the last \IEEEmembership and also \thanks - 
% these prevent an unwanted space from occurring between the last author name
% and the end of the author line. i.e., if you had this:
% 
% \author{....lastname \thanks{...} \thanks{...} }
%                     ^------------^------------^----Do not want these spaces!
%
% a space would be appended to the last name and could cause every name on that
% line to be shifted left slightly. This is one of those "LaTeX things". For
% instance, "\textbf{A} \textbf{B}" will typeset as "A B" not "AB". To get
% "AB" then you have to do: "\textbf{A}\textbf{B}"
% \thanks is no different in this regard, so shield the last } of each \thanks
% that ends a line with a % and do not let a space in before the next \thanks.
% Spaces after \IEEEmembership other than the last one are OK (and needed) as
% you are supposed to have spaces between the names. For what it is worth,
% this is a minor point as most people would not even notice if the said evil
% space somehow managed to creep in.



% The paper headers
\markboth{Journal of \LaTeX\ Class Files,~Vol.~6, No.~1, January~2007}%
{Shell \MakeLowercase{\textit{et al.}}: Bare Demo of IEEEtran.cls for Journals}
% The only time the second header will appear is for the odd numbered pages
% after the title page when using the twoside option.
% 
% *** Note that you probably will NOT want to include the author's ***
% *** name in the headers of peer review papers.                   ***
% You can use \ifCLASSOPTIONpeerreview for conditional compilation here if
% you desire.




% If you want to put a publisher's ID mark on the page you can do it like
% this:
%\IEEEpubid{0000--0000/00\$00.00~\copyright~2007 IEEE}
% Remember, if you use this you must call \IEEEpubidadjcol in the second
% column for its text to clear the IEEEpubid mark.



% use for special paper notices
%\IEEEspecialpapernotice{(Invited Paper)}




% make the title area
\maketitle


\begin{abstract}
%\boldmath
The abstract goes here.
\end{abstract}
% IEEEtran.cls defaults to using nonbold math in the Abstract.
% This preserves the distinction between vectors and scalars. However,
% if the journal you are submitting to favors bold math in the abstract,
% then you can use LaTeX's standard command \boldmath at the very start
% of the abstract to achieve this. Many IEEE journals frown on math
% in the abstract anyway.

% Note that keywords are not normally used for peerreview papers.
\begin{IEEEkeywords}
Wireless Sensor Networks, secure dynamic routing, load-balancing, machine learning, hop-by-hop routing 
\end{IEEEkeywords}






% For peer review papers, you can put extra information on the cover
% page as needed:
% \ifCLASSOPTIONpeerreview
% \begin{center} \bfseries EDICS Category: 3-BBND \end{center}
% \fi
%
% For peerreview papers, this IEEEtran command inserts a page break and
% creates the second title. It will be ignored for other modes.
\IEEEpeerreviewmaketitle



\section{Introduction}
% The very first letter is a 2 line initial drop letter followed
% by the rest of the first word in caps.
% 
% form to use if the first word consists of a single letter:
% \IEEEPARstart{A}{demo} file is ....
% 
% form to use if you need the single drop letter followed by
% normal text (unknown if ever used by IEEE):
% \IEEEPARstart{A}{}demo file is ....
% 
% Some journals put the first two words in caps:
% \IEEEPARstart{T}{his demo} file is ....
% 
% Here we have the typical use of a "T" for an initial drop letter
% and "HIS" in caps to complete the first word.
\IEEEPARstart{W}{ireless} Sensor Networks is a network of sensors that are autonomous. These sensors are spatially distributed for capturing data. They have restricted computational and communication power. They have little memory and limited battery power. The data collected by the individual sensors is then passed on to the base station or the sink. The sink will process the accumulated data for the specific application. Sensors networks have been widely used in military applications, monitoring the environment, healthcare applications and surveillance.\\ \\
\indent In WSN, we have a family of sensors called the \textit{service oriented WSNs}. These WSNs are different from the usual ones, in that they have a specific task. The sensors may not be communicating all the time. They will trigger communication only when they come across a state change. For example, consider a WSN monitoring the enemy infiltration. State change in this situation can be identification of an enemy tank in the surveillance area. State changes in such situations do not occur every now and then. But when they do, it must be reported immediately. A large delay in data delivery or data loss is unacceptable in such situations. \\ \\
\indent In these kind of real time systems, there will be little or room for delay. In such applications congestion detection becomes critical. Effecient load balancing in order to overcome congestion becomes highly important. A packet should  never enter the congested part of the network. Further considering the criticality of the application, it becomes even more important that the data moves along a secure path only. But it is even more important to have a dynamic and adaptive load balancing and security schemes. \\ \\
\indent Dynamic decision making capability at every node will enhance load balancing and security. Handling unforeseen changes in the network will prove beneficial in real time WSNs. Static routing may fail in such cases and lead to packet loss. Whereas handling such situations intelligently at every hop will not only reduce congestion but also prevent packet loss. \\ \\
\indent Therefore it is necessary to have a robust routing technique that is both dynamic and adpative to every change in the network along the path of the packet. Congestion and security analysis can be made dynamic by impelementing them at every hop. This means, we have to check for congestion and security of the next node before sending the packet to that node. When both congestion detection and security monitoring is performed at every node, there is a computation and energy overhead in the network. This will lead to increased latency in the network and will drain the sensor battery quickly. \\ \\
\indent \textit{Motivation}: The resources of a sensor node such as computational power and battery life is limited. But the data transmitted by each of the nodes is of high importance. Most of the existing protocols compromise on congestion detection or security yielding to the limited resources of a sensor. On the other hand, most protocols remain static and do not adapt to the dynamics of the network. Both these classes of protocols will not facilitate the efficient functioning of a service oriented WSN. \\ \\
\indent \textit{Contribution}: In the proposed scheme, every sensor node monitors the load at each of its neighbours. It even analyses the strength of each of its neighbours to determine malicious data. It transmits data only to those nodes that are least congested and highly secure. Since the analysis of the two parameters at every hop introduces an overhead in the network we have a feedback system that enables the network to learn from every decision  taken. Initiatially for a certain part of the communication, the source is made to learn about all the paths that are least congested and most secure through the feedback system. Once the source has sufficient data, it makes an analysis of the collected feedback and predicts the most trustworthy path. A feedback system will inform the node about the rightness of each prediction. \\ \\ 
\indent Our technique basically divides the entire communication into two parts. In the first part, the congestion detection and security analysis is performed and feedback is obtained to justify the decisions taken. In the second part instead of performing the congestion detection and security analysis once again, we use the feedback data already obtained to predict a path for each packet. This will greatly reduce the time and energy that will be spent on analysing the two parameters that happens in the initial part of the communication. \\ \\
\indent The rest of this paper is organized as follows: Section 2 provides a brief review of all the Related Works. The background of the paper is discussed in Section 3. Section 4 defines the main problem and the network model is presented in Section 5. The mathematical model is explained in Section 6 and its implementation is shown in Section 7. The algorithm is discussed in Section 8 and the simulation and performance evaluation is contained in Section 9. The conclusions are presented in Section 10.
% You must have at least 2 lines in the paragraph with the drop letter
% (should never be an issue)
%I wish you the best of success.

%\hfill mds
 
%\hfill January 11, 2007

%\subsection{Subsection Heading Here}
%Subsection text here.

% needed in second column of first page if using \IEEEpubid
%\IEEEpubidadjcol

%\subsubsection{Subsubsection Heading Here}
%Subsubsection text here.


% An example of a floating figure using the graphicx package.
% Note that \label must occur AFTER (or within) \caption.
% For figures, \caption should occur after the \includegraphics.
% Note that IEEEtran v1.7 and later has special internal code that
% is designed to preserve the operation of \label within \caption
% even when the captionsoff option is in effect. However, because
% of issues like this, it may be the safest practice to put all your
% \label just after \caption rather than within \caption{}.
%
% Reminder: the "draftcls" or "draftclsnofoot", not "draft", class
% option should be used if it is desired that the figures are to be
% displayed while in draft mode.
%
%\begin{figure}[!t]
%\centering
%\includegraphics[width=2.5in]{myfigure}
% where an .eps filename suffix will be assumed under latex, 
% and a .pdf suffix will be assumed for pdflatex; or what has been declared
% via \DeclareGraphicsExtensions.
%\caption{Simulation Results}
%\label{fig_sim}
%\end{figure}

% Note that IEEE typically puts floats only at the top, even when this
% results in a large percentage of a column being occupied by floats.


% An example of a double column floating figure using two subfigures.
% (The subfig.sty package must be loaded for this to work.)
% The subfigure \label commands are set within each subfloat command, the
% \label for the overall figure must come after \caption.
% \hfil must be used as a separator to get equal spacing.
% The subfigure.sty package works much the same way, except \subfigure is
% used instead of \subfloat.
%
%\begin{figure*}[!t]
%\centerline{\subfloat[Case I]\includegraphics[width=2.5in]{subfigcase1}%
%\label{fig_first_case}}
%\hfil
%\subfloat[Case II]{\includegraphics[width=2.5in]{subfigcase2}%
%\label{fig_second_case}}}
%\caption{Simulation results}
%\label{fig_sim}
%\end{figure*}
%
% Note that often IEEE papers with subfigures do not employ subfigure
% captions (using the optional argument to \subfloat), but instead will
% reference/describe all of them (a), (b), etc., within the main caption.


% An example of a floating table. Note that, for IEEE style tables, the 
% \caption command should come BEFORE the table. Table text will default to
% \footnotesize as IEEE normally uses this smaller font for tables.
% The \label must come after \caption as always.
%
%\begin{table}[!t]
%% increase table row spacing, adjust to taste
%\renewcommand{\arraystretch}{1.3}
% if using array.sty, it might be a good idea to tweak the value of
% \extrarowheight as needed to properly center the text within the cells
%\caption{An Example of a Table}
%\label{table_example}
%\centering
%% Some packages, such as MDW tools, offer better commands for making tables
%% than the plain LaTeX2e tabular which is used here.
%\begin{tabular}{|c||c|}
%\hline
%One & Two\\
%\hline
%Three & Four\\
%\hline
%\end{tabular}
%\end{table}


% Note that IEEE does not put floats in the very first column - or typically
% anywhere on the first page for that matter. Also, in-text middle ("here")
% positioning is not used. Most IEEE journals use top floats exclusively.
% Note that, LaTeX2e, unlike IEEE journals, places footnotes above bottom
% floats. This can be corrected via the \fnbelowfloat command of the
% stfloats package.

\section{Related Work}
Uluagac et al., \cite{SOBAS} designed a scheme called SOBAS(Secure SOurce-BAsed Loose Synchronization). It securely synchronizes events in the network without transmission of explicit synchronization messages. Nodes use their local time values as one time dynamic key to encrypt each message. Synchronization of events happens quickly and accurately. Loose synchronization reduces the number of control messages required, energy consumption and opportunity for malicious nodes to eavesdrop. High clock precision has not been achieved.\\ \\
\indent Ameer et al., \cite{LeDiR} presented a Least-Disruptive topology Repair (LeDiR) algorithm.  It restores the connectivity without extending the length of the shortest path among nodes compared to the prefailure topology. Implementation involves failure detection, smallest block identification and children movement. It minimizes recovers overhead and maintains shortest path lengths at their prefailure value compared with Recovery through Inward Motion (RIM) and Distributed Actor Recovery Algorithm (DARA). LeDiR is resilient to single node failure at a time. This work can be enhanced for simultaneous node failures.\\ \\
\indent Tao et al., \cite{SKL} introduced mechanisms considering single domain that generate randomized multipath routes. Routes taken by the $shares$ of different packages change over time. Besides randomness, the generated routes are also highly dispersive and energy efficient, making them quite capable of overcoming black holes. The proposed algorithms can be applied to selective packets in WSNs to provide additional security levels against adversaries attempting to acquire these packets. However, adjusting the random propagation and secret sharing parameters, different security levels are achieved at different energy costs.\\ \\
\indent Guoxing et al., \cite{TARF} explore Trust Aware Routing Framework (TARF) that avoids replay attack. When a replay attack has occurred in the network the adversary routes the information in a wrong direction. The TARF algorithm keeps track of trustworthiness of its neighbors and selects the path based on the trust values. It provides high throughput than link-connectivity protocol  and Collection Tree routing Protocol (CTP) in static and dynamic environment respectively. TARF is scalable to medium scale test beds. The protocol needs to be further evaluated with large-scale WSNs deployed in wild environments.\\ \\
\indent Shuai et al., \cite{MASP} proposed a data collection scheme, called Maximum Amount Shortest Path (MASP) for Wireless sensor Networks with path constrained mobile sinks. Genetic algorithm and local search is used to optimize the MASP problem. Sink mobility approaches including MASP and Shortest Path Tree method (SPT) perform better than the static sink approach in terms of energy consumption. This work can be extended for different scenarios with various movement trajectories of mobile sinks, subsink selection problem and security.\\ \\
\indent Xiao et al., \cite{ProHet} introduced a Probabilistic routing protocol for Heterogeneous (ProHet) sensor network protocol. It provides assured delivery rate in a distributed manner using asymmetric links with low overhead. It has two parts: \textit{Preparation part} in which neighbors are identified and a reverse path for an asymmetric link will be find out. \textit{Routing Part} includes selecting nodes, forwarding messages and sending acknowledgement. It provides better delivery ration than AODV. ProHet can be enhanced by incorporating security in it.\\ \\  
\indent Daojing et al., \cite{SDRP} designed  Secure and Distributed Reprogramming Protocol (SDRP). Secure Reprogramming uses identity based cryptography. It minimizes communication and storage requirements. Compared with Elliptic Curve Cryptography (ECC), it provides good solution in terms of key size. Compared with Deluge (Pollution-resistant reprogramming and data dissemination protocol for WSN) and Seluge (Secure and dos-resistant code dissemination protocol for WSN), it achieves more propagation delay and takes less memory than Seluge and more than Deluge.\\ \\
\indent Chonggang et al., \cite{PCCP} proposed a node Priority-based Congestion Control Protocol (PCCP) for Wireless Sensor Networks. Congestion is measured based on packet inter-arrival and service time at each sensor node. Priority index and weighted fairness is applied at each sensor node. It achieves high link utilization and flexible fairness than Congestion Control Fairness protocol (CCF) and small buffer size. It avoids packet loss by achieving lesser delay. It makes use of broadcast nature of wireless channel for forwarding congestion notification packets to the child nodes. PCCP can be further enhanced by incorporating security in it.\\ \\
\indent Bing and Kwan \cite{VOQ} designed a framework for feedback-based scheduling algorithms.  Two sequences of switch configurations are selected and coordinated based on staggered symmetry property and in-order packet delivery property by which efficient feedback mechanism is designed and the packets of same flow experience same delay when passed through different middle-stage Virtual Ouput Queueing (VOQs). It achieves best delay and throughput under various traffic conditions. This work can be extended by incorporating security in it.\\ \\ 
\indent Amir et al., \cite{LBAODV} proposed Load Balancing Ad hoc On-demand Distance Vector Routing (LBAODV) Algorithm. It transmits data simultaneously on all discovered paths. Load is balanced on all discovered paths. It achieves better packet delivery ratio than Ad hoc On-demand Distance Vector(AODV) and longer end-to-end delay than Ad hoc On-demand Multipath Distance Vector (AOMDV). Since more nodes are involved in forwarding data, consumed energies increases simultaneously. Hence, energy is distributed acroos many nodes by increasing the network life time.\\ \\ 
\indent Atieh and Marjan \cite{Survey} have done the survey on various congestion control algorithms: Priority-based congestion Control (PCCP), Queue based Congestion Control Protocol with Priority Support (QCCP-PS), Prioritized Heterogeneous Traffic-Oriented Congestion Control Protocol (PHTCCP), Rate-Controlled Reliable Transport protocol (RCRT), Event to Sink Reliable Transport (ESRT) protocol, Sensor Transmission Control Protocol (STCP).\\ \\
\indent Srikanth et al., \cite{FLARE} proposed Flowlet Aware Routing Engine (FLARE) algorithm. It is a flowlet based splitting algorithm which operates on burst of packets. A flowlet is a burst of packets from a given transport layer flow. Flowlets are characterized by a timeout value, $\delta$. Packet spacing within a flowlet is less than $\delta$. FLARE resides on router that feeds multiple paths and split vector is taken as input which change over time. Tiny overhead incurred by flowlet-based splitting and requires smaller table to track flowlets. This work can be enhanced in terms of security.\\ \\
\indent Maciej and Sarangapani \cite{DPCC} presents a novel, decentralized, predictive congestion control (DPCC) scheme for wireless sensor networks (WSN). It guarantees weighted fairness during congestion by updating weights associated with each packet. It reduces congestion. This work can be extended for real world applications.\\ \\
\indent Lucian et al., \cite{IPS} formulated a scheme consisting of multipath routing protocols, Biased Geographical Routing (BGR) and two congestion control algorithms, named In-network Packet Scatter (IPS) and End-to-End Packet Scatter (EPS). IPS alone incurs smaller throughput penalties for short-range flows and is less beneficial to long-range flows. It gives better throughput when combined with EPS compared to Additive Increase Multiplicative Decrease (AIMD) congestion Control Protocol by incurring a communication overhead of 1 byte per data packet, and has a computational complexity similar to greedy geographic routing.\\ \\
\indent Chieh-Yih Wan et al., \cite{CODA} designed an energy efficient congestion control scheme for sensor networks called CODA (COngestion Detection and Avoidance) that comprises three mechanisms: (i) receiver-based congestion detection, (ii) open-loop hop-by-hop backpressure, and (iii) closed-loop multi-source regulation. Energy tax (Ratio between the total number of packets dropped in the sensor networks and total number of packets received at the sinks over a simulation time.) and fidelity(fidelity is the delivery of the required number of data event packets within a certain time limit.) are the two metrics defined in CODA. It provides minimum fidelity penalty and lower energy tax. This work can be extended by combining with PSFQ(Pump, Slowly, Fetch, Quickly) protocol to achieve better performance.\\ \\
\indent Hye-Young et al., \cite{Hybrid} introduced Hybrid load balancing scheme for games in wireless sensor networks. Dynamic provisioning algorithm based on greedy growing algorithm is combined with  a user-oriented load balancing scheme, which seeks to allocate the heaviest mobile node and nearest mobile nodes in communication with the regions managed by the powerful agents. The occurrence of reallocation for load-balancing will not increase much even though the number of mobile nodes increase. This work can be extended for real world applications.\\ \\
\indent Md. Abdur et al., \cite{Buffer} introduced a congestion detection is based on buffer occupancy of a node. Our congestion detection algorithm is buffer based. On reception of a data packet, each intermediate node monitors its current buffer size (BS) and calculates a running average value using Exponential Weighted Moving Average (EWMA) formula.  When it exceeds the predefined threshold (Th) then the congestion is detected and Congestion notification packet  will be sent to source and sets a flag, which will be cleared after a predefined time interval. This work can be extended by performing analysis task on accuracy and energy efficiency of congestion control mechanisms in wireless sensor networks.\\ \\
\indent Raju Kumaret al., \cite{MCAR} defined MAC-Enhanced (Congestion-Aware Routing) CAR (MCAR), which includes MAC-layer enhancements and a protocol for forming high-priority paths on the fly for each burst of data. CAR requires some overhead for establishing the high-priority routing zone, it is unsuitable for highly mobile data sources. MCAR effectively handles the mobility of high-priority data sources. MCAR’s mechanisms work in a top-down manner. The primary challenge in implementing MCAR involved the strictly modular design of TinyOS.\\ \\
\indent Xiaoyong et al., \cite{LDTS} have implemented Lightweight Dependable Trust System (LDTS) scheme for WSNs. It is based on the nodes’ identities (roles), which is suitable for facilitating energy-saving in clustering algorithms. It cancels the feedback between cluster members (CMs) or between cluster heads (CHs) because of which system efficiency is improved while reducing the effect of malicious nodes by demanding less memory. \\ \\
\indent The Dependability-enhanced trust evaluating approach is adopted by LDTS for cooperations between CHs (Cluster Heads). As the number of clusters increases in the network, the LDTS introduces slightly more storage overhead compared with GTMS (Group-based Trust Management Scheme).\\ \\
\indent Fenye et al., \cite{Trust} introduced a probability model, where multidimensional trust attributes such as subjective trust and objective trust are considered. Subjective trust generated as a result of protocol execution at runtime. Objective trust obtained from actual node status. These trust attributes are applied to trust-based geographic routing and trust-based intrusion detection. This model gives better delivery ratio and less message overhead compared with flooding-based routing and geographical routing. This model can be further enhanced for validating a decentralized trust management for autonomous WSNs without base stations and for more dynamic networks such as mobile WSNs, mobile cyber physical systems, or MANETs (Mobile Ad-hoc Networks).\\ \\
\indent Prajakta et al., \cite{CARAVAN} designed congestion management algorithm called Congestion Avoidance and Route Allocation using Virtual Agent Negotiation (CARAVAN). It involves complete autonomy of VAs (Virtual Agents) where they individually explore the solution space. Every VA exchanges its autonomously calculated route preference information to arrive at an initial allocation of routes. The virtual nature of these deals requires no physical communication and, thereby, reduces communication requirements. In congested network conditions, the performance of CARAVAN improves with the increase in decision points (junctions) available for negotiation. Compared with shortest path algorithms, it gains more travel time.\\ \\
\indent Chiara and Roberto \cite{Model} designed a mathematical model, which allows the evaluation of the statistical distribution of the traffic generated by nodes toward the sink. Their results have shown that how the distribution of traffic changes in time axis when different loads are offered. The model allows the evaluation of the optimum packet size (D$_{opt}$), which maximizes the success probability. The D$_{opt}$ decreases while increasing N(Number of nodes). The key assumption of the model is that all the nodes are synchronized.

\begin{table*}
\begin{center}
\caption{Our Results and Comparison with Previous Results for Secure Routing in Wireless Sensor Networks.}
\begin{tabular}{ |   l    |   l    |   l    |   l    |   l    |   l    | }  \hline 
\label{table:comp}  
\bfseries{Related Work}&\bfseries{Protocol Name}&\bfseries{Concept}&\bfseries{Performance}&\bfseries{Advantages}&\bfseries{Disadvantages}\\\hline   
Shancang \emph{et al.,}& (SM-AODV) Secure   & Path vacant ratio & Higher through and    & Load Balancing, & Security can be still\\   
\cite{SMAODV}, 2013   & Multipath AODV.     & metric is used to & packet delivery ratio,& Congestion control,& enhanced    \\  
                      &                     & evaluate the path & lower end-to-end delay,&secure delivery scheme& as done in  \\
		      &         	    & and link-disjoint & lowest data loss ratio &by using threshold    & Our Paper. \\ 
		      &                     & paths are identified.& compared to AODV.   &secret sharing algorithm.&        \\ \hline   
Ameer \emph{et al.,}  & (LeDiR) Least       & It involves four &   Minimizes recovery  & Does not impose & Resilient to \\    
\cite{LeDiR}, 2013    & Disruptive topology & major steps: Failure& overhead compared  & prefailure    & single node  \\     
                      & Repair algorithm.   & detection, Smallest & with Recovery (RIM)& overhead.    &  failure. \\ 
                      &                     & block identification,&through Inward Motion&             &    \\ 
		      &			    & Replacing faulty node&and Distributed Actor&             &    \\
		      &			    & and children movement.&Recovery algorithm (DARA).&       &    \\\hline  	
Guoxing \emph{et al.,}& (TARF) Trust-Aware  & Secures the multihop& In the presence of & It guards the WSN & Only Replay attacks   \\   
		      & Routing Framework   & Routing by evaluating& misbehaviors,throughput&against the mis-directing&are identified.\\   
                      &                     & the trustworthiness  & is higher than the   & multihop routing, &              \\   
                      &                     & of the neighboring   & link-connectivity   & based on identity &             \\ 
                      &                     & nodes.               & protocol.   & theft(Replay attack).    &             \\ \hline 
Uluagac \emph{et al.,}& SOBAS(Secure SOurce  & It will synchronize& More frequent the packets    &Loose synchroniz-&Low clock \\   
 \cite{SOBAS}, 2013   & BAsed loose          & each node with    & re-encrypted, the smaller the   &-ation need half of&precision,\\   
                      & Synchronization)     & base station      & rate of the False Positive      &the energy needed&Unused\\   
                      &                      &                   & Rate in the network.            &for traditional&Key trail\\  
                      &                      &                   &                                 &schemes,reduced&attempts\\ 
                      &                      &                   &                                 &control messages.&\\ \hline
Our Work              & 		     & At every intermediate& Provides high throughput than   &Avoids congestion,&     \\ 	
2014                  &                      & node, path security  & other algorithms.               & minimizes delay, &      \\	  
                      &                      & is verified and minimum&                                 & provides security.&      \\ 
                      &                      & cost multipaths are &                                 &                 &      \\
                      &                      & selected.          &                                 &                 &    \\ \hline
\end{tabular}
\end{center}  
 \end{table*} 

\section{Background Work}
Shancang et al., \cite{SMAODV} proposed SM-AODV (Secure Multipath AODV) protocol which has an evaluation metric, path vacant ratio, to evaluate and find a set of node-disjoint paths from all available paths. SM-SODV includes three phases.\\ \\
\textit{Phase one: Packet Delivery Scheme}\\ 
\indent In this phase, data is split into multiple data segments by using threshold secret sharing scheme. Data can be recovered from T received packets from a N split packets, then scheme is called (T,N) threshold secret sharing scheme.\\ \\
\textit{Phase Two: Multipath Evaluation and Scheduling}\\
\indent Phase two is important in SM-AODV. IT involves five steps. First, is multipath discovery: all node disjoint paths from source to destination are obtained. Second, Multipath Load Balancing evaluation: Vacant rate of each path is evaluated. Third, using threshold secret sharing scheme, load will be split on multiple paths. Fourth, Based on path vacant ratio, load will be split on multiple paths. Fifth, Congest events will re monitored by congest event module, If congest event occurs, then congest control mechanism will be invoked and load will be load will be forwarded according to congest-level.\\ \\  
\textit{Phase Three: Congestion Control}\\
\indent Three parts are included in this phase. They are congestion detection, Congestion control and notification,  and congestion cancellation and load adjusting. Each node sends Hello message to child nodes to check CONGEST event, If CONGEST event is received, CONGEST\_LEVEL will be verified based on the CONGEST\_LEVEL value 0, 1, 2, 3 as no action is taken, path is normal, paths are congested and action to be taken, paths are congested heavily and multipaths must be adjusted respectively. \\ \\
\indent According to simulation results, SM-AODV has higher packet delivery ratio, throughput, lower end-to-end delay and lowest average data loss ration compared with AODV, AOMDV, CR-AODV. SM-AODV adopts an adaptive congestion control scheme over service oriented architecture, which is effective in node or link failure occurs. This work can be further enhanced in terms of security by including intermediate trustworthiness which is implemented in our paper.



\section{Mathematical Model}

\subsection{Congestion Detection}
In this section, we will discuss a congestion detection model for Wireless Sensor Networks.\\
Consider a node $N$ with packet arrival rate $A_r$ and packet service rate $S_r$
The traffic at node $N$ is given by,
\begin{equation}
T=\frac{A_r}{S_r}
\end{equation}

There will be stability at the node only when the following condition is met,
\begin{equation}
T<1
\end{equation}

Let the $B_s$ be the size of the buffer at node $N$ and $n$ be the total number of neighbors of the node $N$. The probobility that the node is idle at a given instance of time is,
\begin{equation}
\begin{split}
P_{idle} & = \frac{1}{1+(\frac{A_r}{S_r})+{(\frac{A_r}{S_r})}^{2}+{(\frac{A_r}{S_r})}^{3}....{(\frac{A_r}{S_r})}^{\infty}}\\
		 & = \frac{S_r-A_r}{S_r}\\
P_{idle} & = 1-(\frac{A_r}{S_r})
\end{split}
\end{equation}   
From (1) we can say that,
\begin{equation}
P_{idle}=1-T 
\end{equation}
When the number of packets in the buffer is $B_s - N$ we say that the node is tending towards congestion. The probability that the node is tending towards congestion ig given by,
\begin{equation}
\begin{split}
P(B_s - bufferedPackets = N) &=P_{idle}{\Big(\frac{A_r}{S_r}\Big)}^{N}\\
                                               &=\frac{S_r-A_r}{S_r}\times \frac{{A_r}^{N}}{{S_r}^{N}}\\
P(B_s - bufferedPackets = N) &=\frac{(S_r-A_r){A_r}^{N}}{{S_r}^{N+1}}
\end{split}
\end{equation}
Probability that there will be a buffer overflow is given by,
\begin{equation}
P_{overflow}=\sum\limits_{i=B_s}^{\infty}\Big(1-\frac{A_r}{S_r}\Big){\Big(\frac{A_r}{S_r}\Big)}^{i}={\Big(\frac{A_r}{S_r}\Big)}^{B_s}
\end{equation}
In order to prevent packet loss to a certain required value $R_v$the following condition should be met
\begin{equation}
{\Big(\frac{A_r}{S_r}\Big)}^{B_s}\le R_v
\end{equation}
To obtain the buffer size $B_s$ for the required packet loss value $R_v$,
\begin{equation}
\begin{split}
\log{\Big(\frac{A_r}{S_r}\Big)}^{B_s}&\le \log(R_v)\\
B_s \log\Big(\frac{A_s}{S_r}\Big)&\le \log(R_v)\\
B_s &>\frac{\log(R)}{\log\Big(\frac{A_r}{S_r}\Big)}\\
B_s &>\log_\frac{A_r}{S_r}R
\end{split}
\end{equation}
Subsituting (1) in (8),
\begin{equation}
B_s >\log_TR
\end{equation}

\subsection{Node Strength}
Node Strength is the measure of the ability of a node to detect malicious data.\\
Let $K_t$ be the number of true keymatches at a node $N$, then the Node Strength $NS$ of $N$ is,
\begin{equation}
NS\propto \sum K_t
\end{equation}
Further, Node Strength is dependent on the number of false keymatches $K_f$ as,
\begin{equation}
NS\propto \frac{1}{\sum K_f}
\end{equation}
Combining the above two equations Node Strength is given by,
\begin{equation}
NS=k.\frac{K_t}{K_f}
\end{equation}
where $k$ is the constant of proportionality.\\
Node Strength of a node $N$ is a relative parameter, which means that it has meaning only when compared with Node Strengths of other nodes in the same network. As an absolute value, Node Strength has no meaning.\\
Since all the nodes have the same size of keyspace, we can ignore the constant of proportionality $k$,
\begin{equation}
NS=\frac{\sum K_t}{\sum K_f}    
\end{equation}
We know that, probability of a true keymatch is proportional to the total number of true keymatches,
\begin{equation}
P_{K_t}\propto \sum K_t
\end{equation}
\begin{equation}
P_{K_t}=\frac{\sum K_t}{\sum totalKeymatches}
\end{equation}
The propbability of a false keymatch is directly proportional to the total number of false keymatches,
\begin{equation}
P_{K_f} \propto \sum K_f
\end{equation}
\begin{equation}
P_{K_t}=\frac{\sum K_f}{\sum totalKeymatches}
\end{equation}
Therefore we can say that,
\begin{equation}
\frac{P_{K_t}}{P_{K_f}}=\frac{\sum K_t}{\sum K_f}=NS
\end{equation}
Further, we can say that,
\begin{equation}
P_{K_f} = 1-P_{K_t}
\end{equation}
Substituting for $P_{K_f}$,
\begin{equation}
NS=\frac{P_{K_t}}{1-P_{K_t}}
\end{equation}


\section{Secure Dynamic Load Balancing Sceheme}
\subsection{Dynamic Load Balancing}
The congestion detection algorithm basically classifies a node as 'Tending Towards Congestion [TTC]' or 'Available'. A node will be classified as TTC if its buffer is so much filled that it can atmost accomodate only one packet sent by each of its neighbour. A node $n$ with buffer size $B_s$ and a total number of neighboring nodes $N$ will tend towards congestion when the number of packets in its buffer is equal to $B_s - N$. Once the number of packets in the buffer is lesser than $B_s - N$, it will be classified as available.\\ \\
\indent Every node will maintain a \textit{Neighbour Information Table [NIT]}. This table will contain information whether each of its neighbours is tending towards congestion or available. Once a node realizes that it is tending towards congestion, it will imediately send a message to all its neighbours updating them about its status. A node which sends such a message must also send an \textit{available} message to all its neighbours as soon as it comes out of the TTC situation.\\ \\
\indent Once a node receives status information about its neighbours, it will update its \textit{Neighbour Information Table}. When a packet has to be routed, the node will first obtain the multiple paths. Then, the first node on each of these paths is taken and checked for congestion. If the NIT says that the node is available, then the node is retained, else the node gets rejected. 

\subsection{Node Strength}
In this phase, the nodes that clear the congestion detection test will be then tested for node strength. The node strength of a node will tell us the capability of that node to detect malicious data.\\ \\
\indent To determine the node strength of a particular  node, we need to obtain the total number of true keymatches and the total number of false keymatches of that node. Once this data is obtained, the node strength of that node is derived by dividing the total number of true keymatches by the total number of false keymatches. Comparing the node strengths of multiple nodes, we can get a clear picture of the most secure node. The node with the highest node strength will be the most secure node. This means that the node has a good track record of determining legitimate and malicious data.

\subsection{Packet Routing}
Once we select the node with the higest node strength, we can be sure that the node is least congested as well. This is because, the node has been shortlisted for node strength analysis only after analysing its congestion status.\\ \\
\indent Now, the packet will be routed to that node. At the next node the entire secure dynamic route selection procedure is executed to determine the next hop for the packet. By doing so we can be sure that we will be adapting to any of the unforeseen changes in the network that may not be known to the source prior to the routing.

\section{Learning}
\subsection{Route Statistics Collection}
This phase is a training data collection collection phase. It will run parallely from the begining of the data transmission uptil the prediction phase begins. The source node maintains a \textit{Learning Table} that will facilitate the collection of route information. The \textit{Learning Table} consists of a column for the route taken by a packet, one column for the number of packets that went along that route, one column for the sum total of the delay of every packet that went along that route. This is needed to obtain the average delay in future. \\ \\
\indent As soon as  the source node receives an acknowledgement from the destination, it will look for the route information sent by the destination and the timestamp. If the route has not been entered in the \textit{Learning Table} previously, it means that it is a new route taken by a packet. In such a case, a new entery is made to the \textit{Learning Table}. If the route already already exists, the packet count for that entry will be increamented and the sum total of the delay will be updated by determing the packet delay using the timestamp sent by the destination and adding it to the existing value in the corresponding entry of the \textit{Learning Table}. 

\subsection{Weight Assignment}
In this phase, each route will be analysed to determine the best route for a prediction. Once the treshold number of packets have been transmitted, i.e once we have collected sufficient training data, we must analyse each of the routes and assign weights to them. The weight of a route is the trust factor of that route.\\ \\
\indent A route is trust worthy if it has lesser delay and a good number of packets have been sent along that route. A route with least delay alone or a route along which the maximum packets have been sent alone cannot be declared as the most trusted route. Trust factor is a more comprehensive value which evaluates both the delay of the route and the number of packets sent along the route.\\ \\
\indent The weight of a route is determined by dividing the quotient obtained by dividing the average delay along that route by the sum total of the average delay along all the route, by the quotient obtained by diving the number of packets sent along the route by the total number of packets sent by the source.\\ \\  

\subsection{Prediction and Feedback}
The weights of each route will give us an idea about the trust factor of each route. The route with least weight will mean that it has lesser delay and a good number of packets have been sent along that route compared to other routes. We then choose this path to route the next packet. \\ \\
\indent During the prediction phase, the path for a packet will be fixed. This means that at every node the congestion detection scheme and the security scheme need not be executed. This is because we have decided to route the packet along this path only after studying the routes performance until the threshold number of packets were transmitted. Since each node along this route has passed the congestion detection test and the node strength analysis, the performance of the route will reflect in the weight of that route which was considered for deciding to transmit the packet along this route. \\ \\
\indent In order to bypass the congestion detection and node strength schemes, a flag is set in the packet. Further the entire route will be mentioned in the packet before routing it. At every hop, when an intermediate node gets this packet, it checks for the bypass flag. If it is set, the node will blindly route it to the next hop mentioned in the packet. Once the packet reaches the destination, the destination will send the route details and the timestamp of packet arrival back to the source. \\ \\
\indent The source will then compute the delay obtained along the route and repeat the weight assignment procedure and then make the next prediction. This scheme is also dynamic because, if a wrong prediction is made, that means, if the delay along the route increased due to some reason unknown to the source, the feedback from the destination will enlighten the source about the situation. The weight will be accordingly adjusted which will then reduce the trust factor of such routes.

\section{Conclusion}
The conclusion goes here.





% if have a single appendix:
%\appendix[Proof of the Zonklar Equations]
% or
%\appendix  % for no appendix heading
% do not use \section anymore after \appendix, only \section*
% is possibly needed

% use appendices with more than one appendix
% then use \section to start each appendix
% you must declare a \section before using any
% \subsection or using \label (\appendices by itself
% starts a section numbered zero.)
%


%\appendices
%\section{Proof of the First Zonklar Equation}
%Appendix one text goes here.

% you can choose not to have a title for an appendix
% if you want by leaving the argument blank
%\section{}
%Appendix two text goes here.


% use section* for acknowledgement
%\section*{Acknowledgment}


%The authors would like to thank...


% Can use something like this to put references on a page
% by themselves when using endfloat and the captionsoff option.
%\ifCLASSOPTIONcaptionsoff
 % \newpage
%\fi



% trigger a \newpage just before the given reference
% number - used to balance the columns on the last page
% adjust value as needed - may need to be readjusted if
% the document is modified later
%\IEEEtriggeratref{8}
% The "triggered" command can be changed if desired:
%\IEEEtriggercmd{\enlargethispage{-5in}}

% references section

% can use a bibliography generated by BibTeX as a .bbl file
% BibTeX documentation can be easily obtained at:
% http://www.ctan.org/tex-archive/biblio/bibtex/contrib/doc/
% The IEEEtran BibTeX style support page is at:
% http://www.michaelshell.org/tex/ieeetran/bibtex/
%\bibliographystyle{IEEEtran}
% argument is your BibTeX string definitions and bibliography database(s)
%\bibliography{IEEEabrv,../bib/paper}
%
% <OR> manually copy in the resultant .bbl file
% set second argument of \begin to the number of references
% (used to reserve space for the reference number labels box)
%\begin{thebibliography}{1}

%\bibitem{IEEEhowto:kopka}
%H.~Kopka and P.~W. Daly, \emph{A Guide to \LaTeX}, 3rd~ed.\hskip 1em plus
%  0.5em minus 0.4em\relax Harlow, England: Addison-Wesley, 1999.

%\end{thebibliography}

\begin{thebibliography}{00} 
\bibitem{SOBAS}
{{A. Uluagac, R. Beyah and J. Copeland,}}
\newblock {"Secure SOurce-BAsed Loose Synchronization(SOBAS) for Wireless Sensor Networks,"}
\newblock {\it IEEE Transactions on Parallel and Distributed Systems,} vol.24, no. 4, pp. 803-813, April 2013.

\bibitem{LeDiR}
{{Ameer A. Abbasi, Mohammed F. Y. and Uthman A.,}}
\newblock {"Recovering From a Node Failure in Wireless Sensor-Actor Networks with Minimal Topology Changes,"}
\newblock {\it IEEE Transactions on Vehicular Technology,} vol. 62, no. 1 , pp. 256-271, January 2013.

\bibitem{SKL}
{{T. Shu, M. Krunz and S. Liu, }}
\newblock {"Secure Data Collection in Wireless Sensor Networks Using Randomized Dispersive Routes,"} 
\newblock {\it In IEEE Transactions on Mobile Computing,} vol. 9, no. 7, pp. 941-954, July 2010.

\bibitem{TARF}
{{Guoxing Zhan, Weisong Shi and Julia Deng}}
\newblock {"Design and Implementation of TARF:A Trust-Aware Routing Framework for WSNs,"}
\newblock {\it IEEE Transactions on Dependable and Secure Computing,} vol. 9, no. 2, pp. 184-197, 2012.

\bibitem{MASP}
{{Shuai Gao, Hongke Zhang and Sajal K. Das,}}
\newblock {"Efficient Data Collection in Wireless Sensor Networks with Path-Constrained Mobile sinks,"}
\newblock {\it IEEE Transactions on Mobile Computing,} vol. 10, no. 5, pp. 592-608, April 2011.

\bibitem{ProHet}
{{Xiao Chen, Zanxun Dai, WenZhong Li, Yuefei Hu, Jie Wu, Hongehi Shi  and Sanglu Lu,}}
\newblock {"ProHet: A Probabilistic Routing Protocol with Assured Delivery Rate in Wireless Heterogeneous Sensor Networks,"}
\newblock {\it IEEE Transactions on Wireless Communications,} vol. 12, no. 4 , pp. 1524-1531, April 2013.

\bibitem{SDRP}
{{Daojing He, Chun Chen, Sammy Chan  and Jiajun Bu,}}
\newblock{"SDRP: A Secure and Distributed Reprogramming Protocol for Wireless Sensor Networks,"}
\newblock {\it IEEE Transactions on Industrial Electronics,} vol. 59, no. 11, pp. 4155-4163, November 2012.

\bibitem{PCCP}
{{Chonggang Wang, Kazem Sohraby, Victor Lawrence, Bo Li  and Yueming Hu,}}
\newblock {"Priority-based congestion Control in wireless Sensor Networks,"}
\newblock {\it Proceedings of IEEE International Conference on Sensor Networks, Ubiquitous and Trustworthy Computing (SUTC'06),} 2006.

\bibitem{VOQ}
{Bing Hu and Kwan L. Yeung,}
\newblock {"Feedback-Based Scheduling for Load-Balanced Two-Stage Switches,"}
\newblock {\it IEEE Transactions on Networking,} vol. 18, no. 4 , pp. 1077-1090, August 2010.

\bibitem{LBAODV}
{Amir Darehshoorzadeh, Nastooh Taheri Javan, Mehdi Dehghan and Mohammed Khalili,}
\newblock {"LBAODV: A New Load Balancing Multipath Routing Algorithm for Mobile Ad hoc Networks,"}
\newblock {\it Proceedings of IEEE 2008 sixth National Conference on Telecommunication Technologies and IEEE 2008 Malaysia Conference on Photonics,} August 2008.

\bibitem{Survey}
{Atieh Rezaei  and Marjan Kuchaki Rafsanjani}
\newblock {"Congestion Control Protocols in Wireless Sensor Networks: A Survey,"}
\newblock {\it Journal of American Science,} vol. 8, no. 12, pp. 772-777, 2012.

\bibitem{FLARE}
{Srikanth Kandula, Dina Katabi, Shantanu Sinha and Arthur Berger,}
\newblock {"Dynamic Load Balancing Without Packet Reordering,"}
\newblock {\it ACM SIGCOMM Computer Communication Review, NY, USA,,} vol. 37, no. 2, pp. 51-62, April 2007.

\bibitem{DPCC}
{Maciej Zawodniok and Sarangapani Jagannathan,}
\newblock {"Predictive Congestion Control Protocol for Wireless Sensor Networks,"}
\newblock {\it IEEE Transactions on Wireless Communications,} vol. 6, no. 11, pp. 3955-3963, November 2007.

\bibitem{IPS}
{Lucian Popa, Costin Raiciu, Ion Stoica and David S. Rosenblum,}
\newblock {"Reducing Congestion Effects in Wireless Networks by Multipath Routing,"}
\newblock {\it IEEE 14th International Conference on Network Protocols (ICNP), Santa Barbaa(CA),}  pp. 96-105, Nov. 2006.

\bibitem{CODA}
{Chieh-Yih Wan, Shane B. Eisenman and andrew T Campbell,}
\newblock {"CODA: Congestion Detection and Avoidance in Sensor Networks,"}
\newblock {\it ACM, Proceedings of the First International Conference on Embedded Networked Sensor Systems, (SenSys '03), NY, USA}  pp.266-279, November 2013.

\bibitem{Hybrid}
{Hye-Young Kim, Hwa-Jin Park and Seojeong Lee,}
\newblock {"A Hybrid Load Balancing Scheme for Games in Wireless Networks,"}
\newblock {\it International Journal of Distributed Sensor Networks, Hindawi,} pp. 1-7, 2014.

\bibitem{Buffer}
{Md. Abdur Razzaque and Choong Seon Hong,}
\newblock {"Congestion Detection and Control Algorithms for Multipath Data Forwarding in Sensor Networks,"}
\newblock {\it IEEE  11th International Conference on Advanced Communication Technology (ICACT),} pp. 651-653, Feb. 2009.

\bibitem{MCAR}
{Raju Kumar, Riccardo Crepaldi, Hosam Rowaihy, Albert F. Harris, Guohong Cao, Michele Zorzi,  and Thomas F. La Porta,}
\newblock {"Mitigating Performance Degradation in Congested Sensor Networks,"}
\newblock {\it IEEE Transactions on Mobile Computing,} vol. 7, no. 6, pp. 682-697, June 2008.

\bibitem{LDTS}
{Xiaoyong Li, Feng Zhou and Junping Du,}
\newblock {"LDTS: A Lightweight and Dependable Trust System for Clustered Wireless Sensor Networks,"}
\newblock {\it IEEE Transactions on Information Forensics and Security,} vol. 8, no. 6, pp. 924-935, June 2013.

\bibitem{Trust}
{Fenye Bao, Ing-Ray Chen, MoonJeong Chang and Jin-Hee Cho,}
\newblock {"Hierarchical Trust Management for Wireless Sensor Networks and its Applications to Trust-Based Routing and Intrusion Detection,"}
\newblock {\it IEEE Transactions on Network and Service Management,} vol. 9, no. 2, pp. 169-183, June 2012.

\bibitem{CARAVAN}
{Prajakta Desai, Seng W. Loke, Aniruddha Desai, and Jugdutt Singh,}
\newblock {"CARAVAN: Congestion Avoidance and Route Allocation Using Virtual Agent Negotiation,"}
\newblock {\it IEEE Transactions on Intelligent Transportation Systems,} vol. 14, no. 3, pp. 1197-1207, Sept. 2013.

\bibitem{Model}
{Chiara Buratti, and Roberto Verdone,}
\newblock {"Performance Analysis of IEEE 802.15.4 Non Beacon-Enabled Mode,"}
\newblock {\it IEEE Transactions on Vehicular Technology,} vol. 58, no. 7, pp. 3480-3493, Sept. 2009.

\bibitem{SMAODV}
{{Shancang Li, Shanshan Zhao, Xinheng Wang, Kewang Zhang and Ling Li,}}
\newblock {"Adaptive and Secure Load-Balancing Routing Protocol for Service-Oriented Wireless Sensor Networks,"}
\newblock {\it IEEE Systems Journal,} vol.8, no. 2, pp. 1-10, June 2013.

\bibitem{NS3}
{{NS-3 http://www.nsnam.org/}}

%\bibitem{}
%{,  and ,}
%\newblock {","}
%\newblock {\it IEEE Transactions on ,} vol., no. , pp. -, April 2013.


\end{thebibliography}



% biography section
% 
% If you have an EPS/PDF photo (graphicx package needed) extra braces are
% needed around the contents of the optional argument to biography to prevent
% the LaTeX parser from getting confused when it sees the complicated
% \includegraphics command within an optional argument. (You could create
% your own custom macro containing the \includegraphics command to make things
% simpler here.)
%\begin{biography}[{\includegraphics[width=1in,height=1.25in,clip,keepaspectratio]{mshell}}]{Michael Shell}
% or if you just want to reserve a space for a photo:

\begin{IEEEbiography}{Lata B T}
is an Assistant Professor in the Department of Computer Science and Engineering at the University Visvesvaraya College of Engineering, Bangalore University, Bangalore, India. She obtained her B.E in Computer Science and Engineering from Karnatak University, Dharwad and M.Tech degree in Computer Network Engineering from Visveswaraiah Technological University, Belgaum. She is persuing her Ph.D in the area of Wireless Sensor Networks at Bangalore University. Her research interest is in the area of wireless sensor networks. 
\end{IEEEbiography}

% if you will not have a photo at all:
%\begin{IEEEbiographynophoto}{John Doe}
%Biography text here.
%\end{IEEEbiographynophoto}

% insert where needed to balance the two columns on the last page with
% biographies
%\newpage

\begin{IEEEbiographynophoto}{Jane Doe}
Biography text here.
\end{IEEEbiographynophoto}

% You can push biographies down or up by placing
% a \vfill before or after them. The appropriate
% use of \vfill depends on what kind of text is
% on the last page and whether or not the columns
% are being equalized.

%\vfill

% Can be used to pull up biographies so that the bottom of the last one
% is flush with the other column.
%\enlargethispage{-5in}



% that's all folks
\end{document}


